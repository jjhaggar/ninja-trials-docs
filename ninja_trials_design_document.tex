%% LyX 2.0.6 created this file.  For more info, see http://www.lyx.org/.
%% Do not edit unless you really know what you are doing.
\documentclass[spanish]{article}
\usepackage[T1]{fontenc}
\usepackage[utf8]{luainputenc}
\usepackage[a4paper]{geometry}
\geometry{verbose,tmargin=2cm,bmargin=2cm,lmargin=2cm,rmargin=2cm}
\pagestyle{empty}
\usepackage{color}
\usepackage{babel}
\addto\shorthandsspanish{\spanishdeactivate{~<>}}

\usepackage[unicode=true,pdfusetitle,
 bookmarks=true,bookmarksnumbered=true,bookmarksopen=false,
 breaklinks=false,pdfborder={0 0 0},backref=false,colorlinks=false]
 {hyperref}

\makeatletter
\@ifundefined{date}{}{\date{}}
\makeatother

\begin{document}

\title{NINJA TRIALS - DESIGN DOCUMENT }

\maketitle
\begin{center}
\textbf{\textcolor{red}{(1ª Iteración)}}
\par\end{center}

\tableofcontents{}

\pagebreak{}


\section{Introducción }

Ninja Trials es un juego para Android al estilo de la vieja escuela,
con ambientación de temática ninja y humor desenfadado, desarrollado
específicamente para Ouya, utilizando el motor AndEngine.

El juego consiste en varias pruebas simples de habilidad con el mando
(pulsar botones lo más rápido posible, pulsar en el momento exacto,
apuntar con precisión, pulsación rítmica, etcétera). 

Comenzamos a hacer este juego para poner a prueba las capacidades
de AndEngine y para rendir homenaje a grandes juegos de antaño como
Track \& Field, Combat School/Boot Camp, y otros.

Desarrollaremos el juego de forma iterativa y creciente. En las primeras
iteraciones las funcionalidades serán muy limitadas, por lo que hasta
no llegar a un nivel de calidad adecuado no se distribuirá comercialmente.


\section{Características }


\subsection{Características del juego }

Algunas de las características mencionadas son imprescincibles, otras
sólo estarán disponibles en futuras iteraciones. Se marcan con un
asterisco ({*}) las que no necesitan estar presentes en esta iteración.

Selección entre dos personajes, Ryoko (chica ninja) y Sho (chico ninja)
Juego cooperativo simultáneo (en la misma consola, y quizás también
en red) {*} Varios niveles de dificultad (fácil, normal y difícil)
Diferentes endings / finales de juego (en función del personaje y
la dificultad) {*} Controles sencillos (habilidades simples, como
velocidad o precisión de pulsación) No se pueden guardar partidas
(las partidas son cortas e intensas así que no hace falta). Se guardan
las puntuaciones y los logros (en la consola) Se guardan las puntuaciones
y los logros (en la nube) {*} Posibilidad de compartir las puntuaciones
y logros no sólo en el servidor dedicado a Ninja Trials, sino también
en Facebook y similares (incluidas redes sociales dedicadas a juegos)
{*} Programado para Ouya (sin descartar posibles adaptaciones futuras
para cualquier otro sistema Android) usando AndEngine como base Multilingüe
(se detectará el idioma predeterminado del dispositivo, pero también
se deberá permitir elegir luego el idioma en las opciones) Doblado
al japonés (voces de los personajes mientras efectúan las pruebas,
y quizás unos pocos diálogos) {*}


\subsection{Características del motor AndEngine}

AndEngine es un motor de código abierto específico para crear videojuegos
en Android. Su licencia es Apache 2.0, por lo que, mientras demos
el debido crédito, podemos hacer lo que deseemos con nuestro juego.
Más información aquí: http://www.andengine.org/forums/


\subsection{Características de la consola }

OUYA - Procesador Nvidia Tegra 3 T33 - 1GB de RAM - 8GB de almacenamiento
interno (expandible mediante disco duro externo vía USB) - Conexión
HDMI1.4a (acepta resolución 1920x1080 ó 1280x720) - WiFi 802.11 b/g/n
- Puerto Ethernet - Bluetooth LE 4.0 - Un Micro-USB - Un USB 2.0 (ampliable
mediante un Hub de USBs) - Sistema operativo Android 4.1 Jelly Bean
(API Level 16) - Mando inalámbrico con controles estándar y un touchpad
(también acepta otros mandos, como el de la XBox360 con cable) - Capacidad
para conectar hasta cuatro mandos.

Más información aquí: https://devs.ouya.tv/developers/docs


\subsection{Controles del juego}

El mando de Ouya tiene: - Cuatro botones digitales (O, U, Y, y A).
- Un pad direccional digital de cuatro direcciones (D-Pad). - Dos
joysticks analógicos (LS, RS). - Dos botones digitales que se activan
cuando los joysticks son presionados hacia abajo (L3, R3). - Dos gatillos
digitales (L1, R1). - Dos gatillos analógicos (L2, R2). - Un touchpad,
configurado para comportarse como una entrada de ratón (touchpad).
- Un botón de Home (Home).

Más información aquí: https://devs.ouya.tv/developers/docs/controllers

Aclaración: Mientras el equipo no disponga de una OUYA para hacer
las pruebas, mantendremos la opción de manejar el juego con cualquier
teléfono móvil, pero es importante conocer la disposición de botones
en el mando y cómo se controlará el juego en su versión final.

Controles que se utilizarán en el juego:

Mando OUYA Adaptación smartphone Zona de Menús Aceptar Botones “O”,
“U”, “Y”, o el botón “Home” Tocar la opción determinada

Cancelar Botón “A” Botón “Back”

Salir del juego Pulsar 2 veces botón “Home” Pulsar 2 veces botón “Home”

Desplazamiento sobre las opciones “D-Pad”, joysticks analógicos (LS,
RS) Tocar la opción determinada Intros cinemá-ticas (openings) Omitir
cinemática Botón “Home” o cualquiera de los cuatro botones digitales
(O, U, Y, y A) Tocar la pantalla

Finales cinemá-ticas (endings) Mostrar opción de omitir cinemática
Botón “A” Botón “Back”

Confirmar omisión Botón “B” Tocar la pantalla

Cancelar omisión Botón “A” Botón “Back” Zona de pruebas (trials) Arriba/abajo/
derecha/izquierda “D-Pad” “Volume+/Volume-” (sólo se pueden emular
2)

Botón de acción 1 Configurable. Por defecto en los botones “O” y “U”
Tocar la pantalla (mitad izquierda)

Botón de acción 2 Configurable. Por defecto en los botones “Y” y “A”
Tocar la pantalla (mitad derecha)

Botón de acción (cualquiera de los 2 botones de acción) Configurable.
Por defecto en los botones “O”, “U”, “Y” y “A” Tocar la pantalla (cualquier
parte)

Pausa Botón “Home” Botón “Home”

Cambios en la configuración de los botones: En posteriores iteraciones
se podrá alterar la configuración de botones, habiendo tanto configuraciones
predeterminadas como personalizadas. 


\section{Pantallas }

Desglosaremos el juego por pantallas, que es la forma más sencilla
e intuitiva (se mostrarán imágenes con pantallas de ejemplo en https://github.com/jjhaggar/ninja-trials/tree/master/docs/example\_screens
). En la primera iteración habrá 18 tipos de pantallas (si la de \textquotedbl{}Loading\textquotedbl{}
no es necesaria sólo serían 17). ¡Aviso! Se citarán los gráficos músicas
y sonidos escribiendo el nombre del recurso en color azul4.


\subsection{Splash Screen (pantalla de inicio)}

Primera pantalla visible al arrancar la aplicación. Los logos de MadGear
(splash\_logo\_madgear.svg) y Ouya (splash\_logo\_ouya\_color.svg,
aunque este último quizás no) aparecen y desaparecen con un fundido
y modificación del tamaño (y quizás de la posición), mientras están
presentes se escucha el sonido correspondiente a cada uno (sonido
de engranajes con MadGear y la palabra \textquotedbl{}Ouya\textquotedbl{}
con el logo del mismo nombre).

Mientras se muestran los logos (por primera vez) se aprovecha para
cargar todos los recursos/assets del juego y para hacer las conexiones
necesarias vía Internet (aún hay que comprobar si esto es viable,
pues son muchos recursos los que hay que cargar).

La primera vez que se muestran los logotipos, estos no se pueden omitir,
pero las siguientes veces si pulsas un botón pasaríamos al \textquotedbl{}menú
principal\textquotedbl{} (al ya tener los recursos cargados no sería
un problema). 


\subsection{Intro 1 Screen (pantalla de introducción 1)}

De forma sincronizada con la música se mostrará: La silueta de los
ninjas (intro1\_shapes.png) sobre un fondo degradado (intro1\_gradient.jpg).
La palabra \textquotedbl{}NINJA\textquotedbl{} en fuente Impact con
un efecto de blending (intro1\_wordmask\_ninja.svg) con la cara de
Sho (intro1\_sho.png). Movimiento hacia la izquierda. Un par de escenas
de las pruebas intercaladas mientras se ve la palabra \textquotedbl{}NINJA\textquotedbl{}
mencionada en el punto anterior. En principio “correr” y “corte” (se
usarán los assets correspondientes a las pruebas, pero como muestra
se usarán dos imágenes fijas, intro1\_trial\_run.jpg y intro1\_trial\_cut.jpg).
La palabra \textquotedbl{}TRIALS\textquotedbl{} en fuente Impact con
un efecto de blending (intro1\_wordmask\_trials.svg) con la cara de
Ryoko (intro1\_ryoko.png). Movimiento hacia la izquierda. Un par de
escenas de las pruebas intercaladas mientras se ve la palabra \textquotedbl{}TRIALS\textquotedbl{}
mencionada en el punto anterior. En principio “salto” y “shurikens”
(se usarán los assets correspondientes a las pruebas, pero como muestra
se usarán dos imágenes fijas, intro1\_trial\_jump.jpg y intro1\_trial\_throw.jpg).
Aparición del logo \textquotedbl{}NINJA TRIALS\textquotedbl{} (intro1\_logo.png).
Aún por definir tanto el logo como el tipo de aparición (lo ideal
es que sea una aparición \textquotedbl{}especial\textquotedbl{} que
sólo pueda hacerse con ese logo), dicha aparición terminará con la
culminación de la música y un efecto especial de sonido que destaque
y que tiene que estar relacionado con algún elemento del logo, el
sonido de un shuriken al lanzarse y clavarse, por ejemplo (el efecto
que quiero es similar al que tiene la \textquotedbl{}X\textquotedbl{}
al aparecer en la intro de \textquotedbl{}Megaman X\textquotedbl{}
http://youtu.be/J-OKIlcLDgk?t=1m56s ).

Vídeo de ejemplo: Existe un vídeo en estado \textquotedbl{}beta\textquotedbl{}
de la pantalla de introducción, subido en GitHub en el subdirectorio
docs/example\_screen/

https://github.com/jjhaggar/ninja-trials/blob/master/docs/example\_screens/screen\_intro1\_beta.avi

¡Aviso! La música y los gráficos del vídeo son antiguos y sólo sirven
de guía.

Formato animaciones de las escenas de pruebas: Aún no está claro si
las escenas de las cuatro pruebas se harán en “formato vídeo” (utilizando
una secuencia de imágenes pregenerada) o se harán animando los elementos
necesarios de cada prueba (fondos, elementos y sprites). En cualquier
caso hasta no tener terminadas las cuatro pruebas se usarán las 4
imágenes fijas mencionadas.

Si se pulsa un botón (Home, O, U,Y, A) la presentación se interrumpirá
y aparecerá el menú principal. Si no se pulsa ningún botón, al acabar
la presentación aparecerá la pantalla de menú principal. 


\subsection{Main Menu Screen (pantalla de menú principal)}

Se mostrará el logotipo \textquotedbl{}NINJA TRIALS\textquotedbl{}
(menu\_main\_title.png), sobre un fondo \textquotedbl{}baldosado\textquotedbl{}
con patrones de motivos ninjas o japoneses (elegido entre menu\_main\_pattern\_1.png,
menu\_main\_pattern\_2.png y menu\_main\_pattern\_3.png), junto con
tres botones (los botones son textos, no imágenes). El texto de los
botones deberá ser multilingüe (cambiar según idioma de la consola).
El texto que aparezca en el juego como mínimo estará en español e
inglés (en posteriores iteraciones se harán más traducciones).

Cada botón tendrá 3 estados (normal, seleccionado y activado), en
ningún momento habrá a la vez más de uno seleccionado (ni activado).
Para la 1ª iteración los estados serán sencillos (en siguientes iteraciones
se intentará dotar de transiciones entre estados y dar más vistosidad).

Estados de los botones Normal Letras a tamaño normal y de color blanco
(o negras con borde blanco). Seleccionado Letras a tamaño más grande
y de color amarillo (o negras con borde amarillo). Activado Las letras
aumentan paulatinamente de tamaño y el color de las mismas (o de su
borde) parpadea entre rojo y amarillo.

Se podrá cambiar la selección de un botón a otro pulsando izquierda/derecha
en el d-pad de la Ouya. Al cambiar la selección de un botón a otro
se escuchará un sonido (menu\_focus.ogg) y el botón pasará a estado
seleccionado.

Se podrán activar los botones pulsando alguno de los siguiente botones
del mando de la Ouya: “HOME”, “Y”, “O” y “U” (el botón “A” este está
reservado para volver atrás en los menús). Al activar cualquiera de
ellos se escuchara otro sonido (menu\_activate.ogg) y el botón pasará
a estado activado, pasando tras un par de segundos a la pantalla deseada.

Los dos sonidos ( menu\_focus.ogg y menu\_activate.ogg) serán comunes
al resto de menús.

Activar el Botón Game Start (iniciar juego) lleva a la pantalla de
selección de personaje. Activar el Botón Options (opciones) lleva
a la pantalla de opciones. Activar el Botón Achievements (logros)
lleva a la pantalla de logros. Si no se mueve el foco ni se pulsa
ningún botón durante 10 segundos se pasa se pasa al \textquotedbl{}Character
Intro Screen\textquotedbl{} (\textquotedbl{}pantalla de presentación
de personajes\textquotedbl{}). Si se pulsa el botón “A” de la Ouya
se vuelve a la pantalla de inicio (Splash Screen). Si se acciona el
d-pad o se pulsa cualquier botón de la Ouya que no sean “HOME”, “O”,
“U”, “Y” o “A”, se reinicia la cuenta atrás de 10 segundos que nos
llevaría a la pantalla de logros. 


\subsection{Character Intro Screen (pantalla de presentación de personajes) }

Típica pantalla de descripción de los personajes protagonistas, al
estilo \textquotedbl{}Final Fight\textquotedbl{}. A cada vuelta (SplashScreen->Intro1Screen->MenuScreen->CharacterIntroScreen->RecordScreen->Splash...)
se muestra la información de cada personaje sobre su fondo respectivo.
Primero se hace con Ryoko (character\_profile\_ryoko.png y character\_profile\_background\_1.png)
y luego se hace con Sho (character\_profile\_sho.png y character\_profile\_background\_2.png),
y se siguen alternando.

Si se pulsa algún botón (Home, O, U,Y, A) se pasa a la \textquotedbl{}pantalla
de menú principal\textquotedbl{}. Si no se pulsa ningún botón, al
cabo de 5 segundos se pasa a la \textquotedbl{}pantalla de récords\textquotedbl{}
(\textquotedbl{}Record Screen\textquotedbl{}).


\subsection{Record Screen (pantalla de récords)}

Esta pantalla utilizará los datos que previamente (durante la Splash
Screen) se han cargado (si se ha conseguido acceder a Internet) sobre
el listado de récords de los mejores jugadores: listado diario, listado
semanal y listado global (en la 1ª iteración los resultados serán
sólo locales, no se hará ninguna conexión a Internet). Los assets
empleados serán results\_records\_winner\_faces.png (caras de los
personajes) menu\_main\_pattern\_1.png, menu\_main\_pattern\_2.png
y menu\_main\_pattern\_3.png (fondo)

Si no se consigue acceder a Internet o al servidor de los récords,
los datos se sustituirán por un listado de récords almacenado localmente
la última vez que se hizo conexión (aún si es la primera vez que se
juega, habría récords por defecto) y se avisaría al jugador con el
mensaje \textquotedbl{}Sin conexión. Récords no actualizados\textquotedbl{}. 

¡Aviso! En esta iteración el guardado de datos mediante Internet no
está contemplado.

Si se pulsa algún botón (Home, O, U,Y, A) se pasa a la \textquotedbl{}pantalla
de menú principal\textquotedbl{}. Si no se pulsa ningún botón, al
cabo de 5 segundos se pasa al \textquotedbl{}Splash Screen\textquotedbl{}
(se reinicia el bucle). 


\subsection{Option Screen (pantalla de opciones)}

En este menú (en esta iteración) se podrá:

Cambiar el volumen de músicas Cambiar el volumen de efectos de sonido.
Escuchar las músicas. Escuchar los efectos de sonido y voces (soundtest
clásico). Visualizar los controles del juego.

Más adelante se introducirán más opciones.

Los recursos empleados son los siguientes: menu\_options\_volume.png
menu\_options\_controller\_marks.png menu\_options\_controller\_ouya.png
menu\_focus.ogg y menu\_activate.ogg

Al pulsar “HOME” o “A” volveríamos a la pantalla de menú principal.


\subsection{Achievements Screen (pantalla de logros)}

Habrá un total de 35 logros (aunque en la 1ª iteración no será necesario
que funcionan todos ellos). 

La pantalla se dividirá en tres zonas diferenciadas: -Zona de datos
del jugador (arriba a la derecha). -Zona de descripción de logros
(abajo a la derecha). -Zona de iconos (izquierda-centro, ocupando
la mayor parte de la pantalla).

En la zona de datos del jugador se muestran el nombre del jugador
actual y los logros (superados/totales).

En la zona de descripción de logros (menu\_achievements\_container\_description.png
) se muestra un icono grande relacionado con el logro (menu\_achievement\_icons\_big.png),
el nombre y la descripción del mismo. En algunos casos también pueden
mostrarse un marcador de progreso, un sello de “superado” (menu\_achievement\_success\_stamp.png)
y una frase de “prueba superada”.

En la zona de iconos (menu\_achievements\_container\_icons.png) se
mostrará una cuadrícula de 7 filas por 5 columnas con iconos (menu\_achievement\_icons\_small.png)
cuyas imágenes representarán los logros (o mostrarán una interrogación
en el caso de que sean secretos). Estos iconos tendrán dos estados,
normal y seleccionado. Se podrá navegar sobre esta cuadrícula mediante
el cursor (D-Pad ), pulsando arriba abajo izquierda y derecha (al
desplazarnos se oirá menu\_focus.ogg). Si llegamos a la derecha del
todo y pulsamos una vez más derecha, pasaríamos al primero de la izquierda
de esa misma fila, y sucedería lo mismo en el resto de casos (izquierda
, arriba y abajo). Al pulsar sobre un icono se actualiza la zona de
descripción (y se escuchará menu\_activate.ogg).

Tipo ¿Supe-rado? Imagen Descripción Normal No

Muestra un icono descriptivo de la prueba. El color gris del icono
y una imagen de un candado dan a entender que aún no se ha superado.
Si se pulsa sobre él, en la sección de descripción de logros se muestra
el icono detallado, el nombre y la descripción del logro (y en caso
de ser considerado necesario, el progreso actual en ese logro).

Sí

Muestra un icono descriptivo de la prueba. Ahora el icono tiene color
y no hay candado. Si se pulsa sobre él, en la sección de descripción
de logros se muestra el icono detallado y la descripción del logro.
Se muestran también el sello “superado” (menu\_achievement\_success\_stamp.png)
y una frase de éxito sobre el logro. Secreto No

Muestra un icono de una interrogación. El color gris del icono y una
imagen de un candado dan a entender que aún no se ha superado. Si
se pulsa sobre él, en la sección de descripción de logros se muestra
un icono de una interrogación y donde deberían estar el nombre y la
descripción del logro respectivamente se muestra “????” y se avisa
de que no hay descripción disponible. Algunos de los logros secretos
podrán mostrar pistas, ya sea en el título o en la descripción.

Sí

Muestra un icono descriptivo de la prueba. Además el icono tiene color
y no hay candado. Si se pulsa sobre él, en la sección de descripción
de logros se muestra el icono detallado y la descripción del logro.
Se muestran también el sello de “superado” y una frase de éxito relacionada
con el logro.

¡Aviso! Para hacer los placeholders de los iconos, en lugar de hacer
un dibujo descriptivo he añadido números a cada uno de los mismos,
esto se corregirá a medida que se vayan añadiendo los logros.

Al pulsar “HOME” o “A” volveríamos a la pantalla de menú principal.


\subsection{pantalla de selección de personaje }

La selección de personaje y selección de dificultad se hacen en la
misma pantalla, usando el mismo fondo (menu\_select\_sky.png, menu\_select\_clouds.png,
menu\_select\_moon.png, menu\_select\_roof.png). Primero se elige
personaje (menu\_select\_ch\_ryoko.png, menu\_select\_ch\_sho.png)
y luego se elige dificultad (menu\_select\_difficulty.png).

Tanto las opciones de elegir personaje (gráficos) como las opciones
de elegir dificultad (texto) tienen tres estados: normal, seleccionado
y activado. Al cambiar de un estado a otro se escuchan los sonidos
respectivos (o bien el sonido de cambio de selección menu\_focus.ogg
o bien el de activación menu\_activate.ogg).

Para la 1ª iteración los estados serán sencillos (en siguientes iteraciones
se intentará dotar de transiciones entre estados y dar más vistosidad,
con animaciones por ejemplo).

Selección personaje Selección dificultad Normal Sprite normal. Letras
a tamaño normal y de color blanco (o negras con borde blanco). Seleccionado
Sprite amarillo, y sobre él el sprite normal (el sprite amarillo es
ligeramente más grande, así que parecerá que tiene un borde amarillo)
Letras a tamaño más grande y de color amarillo (o negras con borde
amarillo). Activado Sprite amarillo, sobre él el sprite normal, y
sobre ambos de nuevo el sprite amarillo, pero con el valor de transparencia
al 50\% Las letras aumentan paulatinamente de tamaño y el color de
las mismas (o de su borde) parpadea entre rojo y amarillo.

Primero se nos permite elegir entre los personajes (Ryoko y Sho).
Con derecha e izquierda cambiamos entre la selección de uno y otro.
Al pulsar un botón de la Ouya (de entre los siguientes: “HOME”, “O”,
“U” e “Y”) elegimos el personaje que esté seleccionado. Tras elegir
personaje aparecen las opciones de dificultad (Fácil, normal y difícil).
Con arriba y abajo cambiamos entre una y otra. Al pulsar un botón
de la Ouya (de entre los siguientes: “HOME”, “O”, “U” e “Y”) elegimos
la dificultad que esté seleccionada. Tras elegir dificultad pasaremos
a la pantalla de intro 2. Si se pulsa el botón “A” de la Ouya mientras
se está en la selección de dificultad se vuelve a la selección de
personaje. Si se pulsa mientras se está en la selección de personaje,
se pasa al menú principal.


\subsection{Intro 2 Screen (pantalla de introducción 2)}

Se ve al maestro del clan diciéndoles que deben superar las pruebas
para ser considerados ninjas de pleno derecho. El storyboard y los
gráficos están aún por determinar.

Si se pulsa algún botón (Home, O, U,Y, A) se corta la intro y se pasa
a la pantalla de mapa. Si no se pulsa ningún botón, al acabar la introducción
se pasa a la pantalla de mapa.


\subsection{Map Screen (pantalla de mapa)}

En esta pantalla se muestra el texto “Story Mode (Map)” (zona superior
derecha).

También se muestra un mapa (menu\_map\_background.png) con la situación
de las pruebas (puntos marcados menu\_map\_background\_marks.png)
y la posición del ninja (pequeño sprite animado menu\_map\_ch\_ryoko.png,
menu\_map\_ch\_sho.png), que se movería desde la última localización
hasta la de la nueva prueba (excepto si se ha continuado, en cuyo
caso el ninja se quedaría en el mismo sitio). Este mapa estaría como
capa de fondo, ocupando todo el espacio.

Por último, también se muestra un pergamino (menu\_map\_scroll.png)
que se desenrolla (en diferentes posiciones, según la prueba, para
no tapar la localización actual del ninja). Tras desenrollarse, aparece
sobre el pergamino una imagen de la prueba (menu\_map\_drawings.png)
que debe hacerse en este momento (en esta iteración el jugador no
decide nada en el mapa, deberá hacer las pruebas por orden).

Imágen Descripción

El personaje constará de dos animaciones en el mapa:

1) Standing: El personaje está quieto, la animación es su respiración
(usar por ejemplo los dos frames superiores). 

2) Walking: El personaje camina, lo hace mirando sólo hacia abajo,
independientemente de hacia donde se dirija (usar por ejemplo los
dos frames inferiores).

Los estados de las marcas del mapa son, por orden:

1) Aún no disponible.

2) Seleccionada (estamos sobre esta marca).

3) Activada (estaba seleccionada y hemos pulsado el botón).

4) Completada.

¡Aviso! El número de fotogramas de las animaciones de los placeholders
de los personajes podría cambiar (y no sólo en esta pantalla, también
en el resto), aunque siempre se intentará que se asemejen lo más posible
a los spritesheets finales. El número de animaciones de los placeholders
en cambio sí que deberían coincidir con los de los spritesheets finales
(salvo en el caso de algún extraño imprevisto).

Si no se pulsa ningún botón, al pasar 20 segundos se pasa a la pantalla
de Cómo jugar. Si se pulsa algún botón (Home, O, U,Y, A) se pasa directamente
a la pantalla de Cómo jugar.


\subsection{How to Play Screen (pantalla de cómo jugar)}

Se mostrará una escena con fondo neutro (probablemente negro) y tres
zonas diferenciadas: - Zona de imagen de gameplay (izquierda de la
pantalla) - Zona de controles (derecha de la pantalla) - Zona de descripción
de la prueba (parte inferior de la pantalla)

En la zona de imagen de gameplay se mostrarán a tamaño reducido (50\%
aproximadamente) los elementos necesarios para la prueba que se desea
explicar (utilizando los assets propios de la prueba) dichos elementos
son el personaje, hud y elementos relacionados, el fondo sólo se mostrará
si es absolutamente necesario.

En la zona de controles se mostrarán los controles que se utilizarán
durante la prueba, y marcas (how\_to\_play\_arrow.png) que indicarán
cuando utilizarlos. Los controles que se podrán mostrar son el D-pad
(how\_to\_play\_digital\_pad.png) y los botones de acción (how\_to\_play\_button.png).

En la zona de descripción de la prueba se mostrará un texto con una
breve explicación de lo que debe hacer el jugador en la prueba concreta.
El texto puede cambiar a medida que se explican las diferentes acciones
que puedan hacerse en dicha prueba. También se explicará la utilidad
de los indicadores HUD.

Si no se pulsa ningún botón al pasar 5 segundos se pasa a la pantalla
de juego. Si se pulsa algún botón (Home, O, U,Y, A) se pasa directamente
a la pantalla de juego.


\subsection{Gameplay Screen (pantalla de juego)}

Hay varios tipos de juego (trials). Para la primera iteración se planea
hacer cuatro diferentes, \textquotedbl{}Carrera\textquotedbl{}, \textquotedbl{}Corte\textquotedbl{},
\textquotedbl{}Salto\textquotedbl{} y \textquotedbl{}Lanzamiento de
shurikens\textquotedbl{}. 

¡Aviso! Aunque en algunas imágenes se muestre que hay dos jugadores
en pantalla, no se requieren dos jugadores simultáneos para esta iteración.

Resumen de puntuaciones en las pruebas que se incluirán en esta iteración:
Prueba Común a todas las pruebas Específico de cada prueba.

Superar prueba (10.000 puntos) Great (5.000 puntos) Varios (puntos
variables, se calcularán tras probar los “trials”) Perfect (puntuación
x2) Correr Recorrer distancia antes de que se agote el tiempo total
Superar prueba con suficiente tiempo extra Máxima velocidad alcanzada

Cantidad de segundos acumulados en “velocidad máxima” Toda la prueba
a “velocidad máxima”

No existe una velocidad máxima tope. Existe una velocidad máxima que
define si estás a “velocidad máxima” (sirve para el hi-speed combo).
Existe también la velocidad máxima real alcanzada (pulsaciones por
unidad de tiempo), que sólo depende de lo rápido que pulse el jugador.
Max.vel.alcanzada --> Seg.acum.vel.max. --> Corte Cortar dentro del
umbral de precisión antes de agotar el tiempo Superar prueba con una
precisión mínima Precisión conseguida Tiempo sobrante Precisión perfecta

Precisión conseguida --> Tiempo sobrante --> Salto Llegar a la altura
requerida antes de agotar el tiempo Superar prueba con suficiente
tiempo extra Cantidad de saltos perfectos Tiempo sobrante Todos los
saltos perfectos

Saltos perfectos --> Tiempo sobrante --> Shurikens Eliminar el total
de objetivos antes de recibir un número de ataques Superar prueba
con una precisión mínima Precisión conseguida Tiempo sobrante acumulativo
de eliminación de objetivos Precisión perfecta

Precisión conseguida --> Tiempo sobrante --> 

Algunas pruebas que se podrían introducir en posteriores iteraciones
son: Prueba Descripción Meditación en la cascada Pulsar direcciones
y botones de acción según se indique en pantalla para esquivar peces,
troncos y otros objetos que arrastra la corriente (de forma similar
a los juegos musicales/bemani, como StepMania) Carrera de saltos Pulsar
rápido el botón de acción 1 para correr y cada vez que haya que saltar
pulsar el botón de acción 2 Carrera de obstáculos Pulsar rápido el
botón de acción y cada vez que haya que esquivar un obstáculo diferente
pulsar la direccion del D-Pad correspondiente Lanzamiento de ¿lanza?
Pulsar rápido botón de acción 1 para coger carrerilla hasta el borde
de un barranco, al llegar al borde pulsar el botón de acción 2 y dejarlo
pulsado mientras se consigue el ángulo adecuado, soltar en el momento
justo. Tiro con arco (distancia) Mantener pulsado el botón de acción
hasta cargar el arco (sin pasarse) y luego pulsar el botón en el momento
justo para ajustar el ángulo. Tiro con arco (precisión) Pulsar el
botón de acción (cualquiera) e intentar apuntar a la diana (contrarrestando
el peso del arco y el cambio del viento) y soltar el botón en el momento
que el punto de mira está en el centro de la diana (¿como podríamos
hacer esto? los arcos no tienen punto de mira) Lanzamiento al aire
y corte de objetos (troncos) antes de que caigan Pulsar el botón en
el momento justo (similar al corte de velas, pero en este caso la
barra se va moviendo cada vez más rápido y cuanto más cerca del final
se pulse el botón, más puntos se consiguen. Si se falla o se pasa
el tiempo, te cae alguno de los pequeños troncos en la cabeza (o el
adversario gana el duelo). Duelo samurai con espadas de bambú

Carrera a nado Pulsar rápida y alternadamente el botón de acción 1
y el botón de acción 2 (o izquierda y derecha en el D-Pad)

A continuación se definirán cada una de las pruebas (trials) del siguiente
modo: Una breve descripción de la prueba. Una tabla con la cabecera
en color salmón que explicará los marcadores (HUD). Se mencionarán
los recursos (assets) del HUD. Una tabla con la cabecera en color
azul cielo que explicará los controles. Una tabla con la cabecera
en color gris azulado que explicará cómo se puntuará en la prueba.
Una tabla con la cabecera en color cyan 10 que explicará qué cambiará
al variar la dificultad. Una tabla con la cabecera en color amarillo
verdoso 2 que describirá con más detalle el desarrollo de la prueba.
Aquí se mencionarán los recursos (assets) del personaje, del fondo
y de los objetos de la prueba. Una tabla en color amarillo 10 que
explicará cómo se gana o se pierde en la prueba. Una tabla con la
cabecera en color naranja 1 que mostrará y explicará brevemente la
utilización de los assets (música, sonido y gráficos) empleados en
la fase

Si en cualquier prueba, antes de su conclusión se pulsa el \textquotedbl{}Start\textquotedbl{}
se pasará a la \textquotedbl{}pantalla de pausa\textquotedbl{}.


\subsubsection{Run Trial (prueba de carrera)}

Esta prueba consiste en pulsar rápidamente el botón de acción para
que aumente la energía del personaje, y que gracias a ella el personaje
recorra una distancia establecida en un límite de tiempo establecido.

Algo similar a la fase de los 100m lisos de Track \& Field (© Konami):
http://youtu.be/RLeNExXflkc?t=20s

Marcadores Descripción Cuenta atrás Al inicio se mostrará una cuenta
atrás (Ready? 3, 2, 1, Go!). El jugador puede pulsar durante la misma,
eso hará que su energía aumente y pueda salir disparado al dar la
señal de salida. Cronómetro Situado en la esquina superior derecha,
empezará a correr tras la salida de la cuenta atrás. Barra de energía
La barra (power\_bar\_push.png) muestra la energía actual del personaje
entre 0 y 100 (hay un pequeño porcentaje de energía extra que no se
mostraría en la barra). Cuanto más llena esté, más correrá el personaje.
Cara del personaje La cara (ch\_head\_run.png) mostrará la energía
actual del personaje (aunque de forma menos exacta) mostrando un fotograma
u otro (1=quieto, 2=corriendo y 3=corriendo rápido). Localizador Barra
mostrada en la parte superior (run\_line\_bar.png). Indica la posición
actual del jugador en el recorrido mediante una marca (run\_line\_mark.png).
Combo de máxima velocidad Si el jugador mantiene una energía mayor
a cierto porcentaje (alrededor del 90\%) a cada instante que pase
conseguirá puntos extra y se mostrará en pantalla un texto que le
informe del número de segundos (con 2 decimales) que lleva corriendo
a máxima velocidad

Controles: Sólo hay que pulsar un botón lo más rápido posible. - Cuanto
más rápido se pulse el botón de acción más se aumentará la energía
(reacciona a cualquier botón de acción, pero hasta que no se levante
la pulsación del botón pulsado no se registrará la siguiente pulsación
de otro botón, para evitar el \textquotedbl{}truco\textquotedbl{}
de pulsar todos los botones del mando a la vez). - La energía se reflejará
en la animación del personaje, la barra de energía y la cara del personaje
en el HUD. - Al pasar el tiempo se irá decrementando la energía, por
lo que si el jugador deja de pulsar rápido, la barra de energía descenderá.

La puntuación en la prueba la determinarán las siguientes mediciones:
- Tiempo en recorrer la distancia. - Max Speed Combo. Tanto el máximo
combo continuado como el total acumulado. - Máxima velocidad alcanzada.

Los niveles de dificultad implicarán diferencias en: - Distancia a
recorrer. - Tiempo máximo para completar el recorrido. - Decremento
de energía por unidad de tiempo. 

Descripción detallada Prepa-ración Se muestra el escenario de correr
(run\_background\_trees\_front.png, run\_background\_floor.png, run\_background\_trees\_back.png
todos ellos moviéndose con efecto parallax horizontal), al árbitro
preparado para dar la salida (run\_obj\_judge.png) y al personaje
preparándose para iniciar la carrera (run\_ch\_ryoko.png y run\_ch\_sho.png). 

El HUD muestra en la esquina inferior izquierda la cara del personaje
(hud\_head\_run.png) y la barra de energía (hud\_power\_indicator.png),
y en la superior derecha el cronómetro (detenido). En la zona superior
de la pantalla se muestra la barra del localizador (run\_line\_bar.png)
con la marca (run\_line\_mark.png) a la izquierda del todo.

Se muestra en el centro de la pantalla un aviso de preparación/cuenta
atrás (Ready? 3, 2, 1, Go!). El jugador puede pulsar el botón de acción
durante la misma, eso hará que su energía aumente y de ese modo pueda
salir disparado (desprendiendo una polvareda run\_dust\_start.png)
al dar la señal de salida (durante la preparación la energía no disminuye
con el paso del tiempo). Juego Tras el inicio del juego se pone en
marcha el cronómetro (realmente el jugador ya debería haber empezado
a pulsar antes). La cara del personaje en el HUD cambia según el valor
de energía en ese momento. 

Dependiendo de lo rápido que se esté pulsando, se mostrará una animación
u otra: -Detenido (energía == 0) --> El ninja se detiene o no ha empezado
a correr -Corriendo (energía >= 0 \&\& energía <= 50\%) --> El ninja
está corriendo, pero a poca velocidad -Corriendo rápido (energía >
50\%) --> El ninja corre muy rápido, y a medida que corre va desprendiendo
una polvareda (run\_dust\_continuous.png).

Tras recorrer la distancia requerida (o al acabarse el tiempo) finaliza
la parte jugable de la prueba. Resultado Dependiendo de si se ha superado
o no la prueba, se mostrará la animación de victoria o la animación
de derrota. 

En la línea de llegada se encontrarán otro árbitro idéntico al de
salida y un torii (run\_bg\_torii.png) que hará de línea de llegada.

Victoria: Se recorre la distancia antes de que acabe el tiempo. Se
pasa a la pantalla de \textquotedbl{}victoria/resultados\textquotedbl{}.
Derrota: Se acaba el tiempo del cronómetro antes de que se logre recorrer
la distancia necesaria. Se pasa a la pantalla de \textquotedbl{}derrota
/ continuar\textquotedbl{}

Recursos (assets) En estos spritesheets se encuentran todas las animaciones
de los personajes en este trial.

Preparation: Se usa cuando el personaje aún no ha empezado a correr
o cuando se para al dejar de pulsar el botón. Run normal: Se usa cuando
el personaje está corriendo pero no va a más de un porcentaje concreto
de la máxima velocidad (por ejemplo 75\%) Run fast: Se usa cuando
el personaje está corriendo a más de un porcentaje concreto de la
máxima velocidad (por ejemplo 75\% o más) Win: Se usa cuando se acaba
la prueba con éxito. Lose: Se usa cuando se acaba el tiempo antes
de llegar a la meta.

run\_ch\_ryoko.png (run\_ch\_sho.png)

Este spritesheet se usa para la animación del polvo que se levanta
del suelo al iniciar la carrera si el personaje arranca con una energía
superior al 50\% (por poner un número). run\_dust\_start.png

Este spritesheet se usa para la animación del polvo que se levanta
del suelo mientras el personaje corre con una energía superior al
50\%. run\_dust\_continuous.png

Una de las tres capas principales del fondo parallax.

Se encontrará situada al fondo del scroll.

run\_background\_trees\_back.png

Una de las tres capas principales del fondo parallax.

Se encontrará situada sobre la capa del fondo del scroll parallax.

run\_background\_trees\_front.png

Una de las tres capas principales del fondo parallax.

Se encontrará situada sobre las otras dos del scroll parallax.

run\_background\_floor.png

Un elemento animado del fondo parallax. 

Aparecerá al principio y al final del recorrido (como juez de salida
y llegada)

Se encontrará situado sobre las tres capas del fondo del scroll parallax.
run\_obj\_judge.png

Un elemento fijo del fondo parallax. 

Aparecerá al final del recorrido. 

La imagen debe ser separada en dos por la marca blanca vertical discontinua
, para que la mitad izquierda se sitúe bajo la capa de personajes
y la mitad derecha se sitúe sobre los personajes. run\_bg\_torii.png

Un elemento del HUD.

Se mostrará un sprite u otro dependiendo de la energía del personaje.
hud\_head\_run.png

Un elemento del HUD.

Se mostrará la anima-ción debajo y sobre la cabeza del hud con un
alpha de 0'5 cuando la energía del personaje esté al máximo. hud\_heads\_aura.png

Un elemento del HUD.

Sobre la zona del degradado se situará un rectángulo tapando de arriba
abajo, dejando mostrar más área de la imagen cuanta más energía tenga
el personaje. hud\_power\_indicator.png

Un elemento del HUD.

Servirá para indicar la situación del personaje en la carrera. run\_line\_bar.png

Un elemento del HUD.

Se dividirá en 2 trozos, uno para cada jugador.

Se situará sobre la línea “run\_line\_bar.png” indicando la situación
actual del personaje en la carrera. run\_line\_mark.png

Música:

Nombre de archivo: trial\_run.ogg

Loop point: 1 segundo y 600ms Sonidos:

Especificados en el epígrafe 4.2) Efectos de sonido


\subsubsection{Cut Trial (prueba de corte)}

Esta prueba consiste en pulsar un botón en un instante exacto indicado
por un marcador de concentración para que el personaje corte las mechas
de un grupo de velas encendidas y consiga que se apaguen sin romperlas.

Algo similar a la fase de bonus de bottle cut del Art of Fighting
(© SNK): http://youtu.be/p4ysiQ1IIcA?t=9m42s http://youtu.be/p4ysiQ1IIcA?t=16m34s

Marcadores Descripción Cuenta atrás Al inicio se mostrará una cuenta
atrás (Ready? Go!). El jugador no puede pulsar durante la misma. Cronómetro
Situado en la esquina superior derecha, empezará a correr tras la
salida de la cuenta atrás. Barra de concentración La barra (hud\_precision\_indicator.png)
tiene varios rangos de marcas diferenciadas por colores que se distribuyen
respecto al centro de la misma, y una marca/flecha móvil (hud\_precision\_cursor.png)
que oscila de derecha a izquierda a una velocidad constante. Cuanto
más cerca esté la flecha del centro de la barra, más concentrado estará
el personaje (la barra irá de 0 a 100, siendo 50 el centro exacto).
Cara del personaje Mostrará diferentes animaciones (de pocos fotogramas)
indicando la concentración. Una vez se pulse el botón mostrará otra
animación correspondiente al corte (ambas animaciones estarán en hud\_head\_cut.png).

Controles: Sólo hay que pulsar un botón de acción una vez en el momento
justo. - A medida que pasa el tiempo la flecha móvil de la barra de
concentración irá oscilando de derecha a izquierda. - El tiempo del
cronómetro irá descendiendo. - Al pulsar el botón, se muestran los
ojos del ninja en primer plano y justo después se muestra la animación
del corte. Por último se ve el resultado del corte (para dar un poco
de emoción se oculta el marcador de concentración hasta que se ve
el resultado). - Cuanto más cerca del centro esté la flecha en el
momento de la pulsación, mejor será la concentración y mejor será
el corte (las animaciones del resultado cambian según la concentración).

La puntuación en la prueba la determinarán las siguientes mediciones:
- Concentración (cuanto más cerca de la marca del centro, más concentración
y más puntos). - Rapidez al pulsar (la máxima puntuación se consigue
pulsando el botón en el momento exacto en el centro de la primera
mitad de la oscilación de la flecha).

Los niveles de dificultad implicarán diferencias en: - Tiempo máximo
para realizar el corte. - Velocidad de oscilación de la flecha. 

Descripción detallada Prepa-ración Se muestra el escenario de corte
(cut\_background.png), con los elementos a cortar, que son las velas
(cut\_breakable\_candle\_base.png, cut\_breakable\_candle\_light.png)
y un trozo del fondo (cut\_breakable\_tree.png) y al personaje preparándose
para hacer el corte (en los placeholder actuales, el primer fotograma
de cut\_ch\_ryoko\_cut\_anim.png y cut\_ch\_sho\_cut\_anim.png). El
HUD muestra en la esquina inferior izquierda la cara del personaje
y la barra de concentración, y en la superior derecha el cronómetro,
todos ellos aún inactivos. Se muestra en el centro de la pantalla
un aviso de preparación/cuenta atrás (“Ready?” “Cut!”). Juego Tras
el inicio del juego se ponen en marcha el cronómetro y la barra de
concentración/precisión. La cara del personaje en el HUD cambia según
el valor de concentración en ese momento. Tras pulsar el botón (o
al finalizarse el tiempo) primero se ve un zoom de los ojos del personaje
(cut\_ch\_ryoko\_eyes.png, cut\_ch\_sho\_eyes.png) y después se muestra
una animación (los 3 últimos fotogramas de cut\_ch\_ryoko\_cut\_anim.png
y cut\_ch\_sho\_cut\_anim.png) en la que realiza los cortes (cut\_sword\_sparkle1.png
, cut\_sword\_sparkle2.png), tras eso hay una pausa dramática (finaliza
la parte jugable de la prueba). Resultado Dependiendo de lo bien que
se haya realizado la prueba, se mostrará una animación u otra y un
texto (el texto es opcional, cuando tengamos la pantalla de resultados
no hará falta mostrarlo) informando de la puntuación obtenida. Los
resultados posibles son: Fallo --> El ninja no corta las velas, sino
que les da un golpe con la espada, unas se caen y otras salen volando.
Las velas incendian algún elemento del escenario (esto no se muestra
en esta iteración). El ninja muestra una animación de derrota cómica
típica del manganime ( permanece en la misma posición de corte y le
cae una gota de sudor gigante por la cabeza cut\_ch\_sweatdrop.png).
Normal --> Corta las velas, que caen resbalando a la altura del corte
y se apagan. Algunas se tambalean. El ninja pone su pose ganadora
(en la primera iteración el ninja permanece en la misma posición de
corte y le brilla uno de los ojos cut\_ch\_sparkle.png). Bien -->
Corta las mechas de las velas y el fuego de las mismas se apaga. El
ninja pone su pose ganadora. Perfecto --> Corta el fuego las mechas
de las velas. También corta un elemento del escenario (aún por decidir,
en el placeholder actual es un árbol). El ninja pone su pose ganadora
especial.

¡Aviso! En la 1ª iteración sólo se mostrará una animación para fallo
(gota de sudor) y otra para el resto de resultados (brillo en la mirada)
y el texto informando la puntuación (el texto es opcional, cuando
tengamos la pantalla de resultados no hará falta mostrarlo). 

Victoria: Pulsar el botón dentro de los rangos de marcas del centro
de la barra. Se pasa a la pantalla de \textquotedbl{}victoria / resultados\textquotedbl{}.
Derrota: Se acaba el tiempo antes de que se pulse el botón o se pulsa
pero fuera de los rangos de las marcas centrales. Se pasa a la pantalla
de \textquotedbl{}derrota / continuar\textquotedbl{}

Recursos (assets) En estos spritesheets se encuentran todas las animaciones
de los personajes en este trial.

Preparation: El personaje aún se está preparando para hacer el corte
(aunque ahora sólo usamos un frame, se planea aumentar el número).

Cut animation: El personaje realiza el corte y permanece en la postura
del último fotograma.

Win: El personaje permanece en el último fotograma de corte y le brilla
un ojo.

Lose: El personaje permanece en el último fotograma de corte y le
cae una gota de sudor por la frente. cut\_ch\_sho\_cut\_anim.png (cut\_ch\_ryoko\_cut\_anim.png)

Estas imágenes se usan como efectos de las animaciones de victoria
(brillo de un ojo) y de derrota (gota de sudor cayendo por la frente)
respectivamente. cut\_ch\_sparkle.png / cut\_ch\_sweatdrop.png Imagen
usada para semejar un zoom a los ojos del personaje al cortar.

cut\_ch\_sho\_eyes.png (cut\_ch\_ryoko\_eyes.png)

Brillos de la espada mientras se hacen los cortes.

La primera es sólo una imagen, la segunda es una animación.

cut\_sword\_sparkle1.png / cut\_sword\_sparkle2.png 

Fondo de la fase. Ahora mismo es fijo y en una sola capa. cut\_background.png

Cuerpo del elemento a cortar. Ahora mismo son dos candelabros, pero
se planean hacer velas. Se incluiría las animaciones de las velas
cortadas. cut\_breakable\_candle\_base.png

Animación con la luz de los candelabros. Se planea que incluya las
animaciones de las llamas de las velas, tanto mientras arden como
mientras se apagan.

cut\_breakable\_candle\_light.png

Cuerpo de un elemento del fondo que se corta sólo cuando se hace un
corte con una puntuación excepcional.

Ahora mismo es un árbol. Se podría alterar o sumar más elementos.
Incluiría también la animación de este elemento cuando es cortado

cut\_breakable\_tree.png

Este elemento del HUD no se usa en esta iteración. hud\_head\_cut.png

Este elemento del HUD servirá para indicar la concentración actual
del personaje. hud\_precision\_cursor.png

Este elemento del HUD se usa junto con el cursor (el anterior), que
señala oscilando sobre este. 

La máxima concentración sería cuando el cursor marca sobre la línea
central. hud\_precision\_indicator.png

Música:

Nombre de archivo: trial\_cut.ogg

Loop point: - Sonidos:

Especificados en el epígrafe 4.2) Efectos de sonido


\subsubsection{Jump Trial (prueba de salto)}

Esta prueba consiste en subir botando entre dos cañas de bambú increíblemente
altas. Hay que conseguir llegar a una altura especificada antes de
un tiempo especificado. Para ello hay que pulsar un botón en un instante
exacto indicado por un marcador de salto que aparece cada vez que
el personaje intenta rebotar en un bambú. 

Algo similar a la fase de tower topper del Numan Athletics (© Namco):
http://youtu.be/nhwtq4chhlI?t=1m13s

Marcadores Descripción Cuenta atrás Al inicio se mostrará una cuenta
atrás (Ready? Go!). Las pulsaciones del jugador no tienen efecto durante
la misma. Cronómetro Situado en la esquina superior derecha, empezará
a correr tras la salida de la cuenta atrás. Marcador de salto El marcador
(hud\_angle\_indicator.png) tiene varios rangos diferenciados por
colores que se distribuyen según el ángulo, sobre ella hay una marca
y una flecha móvil (hud\_angle\_cursor.png) que se desplazan hacia
arriba una vez el personaje se pose sobre el bambú. Cuanto más cerca
esté la flecha de la marca celeste (que está dentro de la zona roja)
de la barra, mejor será el ángulo y más saltará el personaje. Cara
del personaje Mostrará diferentes animaciones de la cabeza (hud\_head\_jump.png)
del personaje, con pocos fotogramas (en el placeholder actual, de
un sólo fotograma) para indicar el estado del salto (rebotando / en
el aire). 

Controles: Hay que pulsar un botón repetidamente en el momento justo.
- Tras la cuenta atrás el tiempo del cronómetro irá descendiendo.
- El primer salto (desde el suelo) se hace simplemente al pulsar un
botón. - Cuando el personaje toca el bambú que tiene enfrente, la
flecha móvil de la barra de salto irá moviéndose hacia arriba, formando
la línea que va siguiendo a la flecha un ángulo cada vez mayor. El
punto óptimo está marcado por una línea azul. - Si la flecha móvil
asciende demasiado, el personaje se caerá. - Si el personaje cae desde
menos de cierta altura (por ejemplo, dos saltos), la caída no hace
perder la prueba, sino que se puede continuar saltando (el cronómetro
no se resetearía). - Si el personaje cae desde más arriba de esa altura,
se pierde la prueba.

La puntuación en la prueba la determinarán las siguientes mediciones:
- Tiempo en llegar arriba. - Cantidad de saltos perfectos. - Máximo
combo de saltos perfectos.

Los niveles de dificultad implicarán diferencias en: - Tiempo máximo
para llegar arriba. - Velocidad de movimiento en el indicador del
marcador de salto. - Precisión necesaria para conseguir un salto perfecto. 

Victoria: Llegar arriba del todo antes de que acabe el tiempo del
cronómetro. Se pasa a la pantalla de \textquotedbl{}victoria / resultados\textquotedbl{}.
Derrota: Se acaba el tiempo antes de que se llegue arriba o el personaje
se cae (desde suficiente altura) debido a que pulsa el botón en mal
momento (o no lo pulsa). Se pasa a la pantalla de \textquotedbl{}derrota
/ continuar\textquotedbl{}

Descripción detallada Prepa-ración Se muestra el escenario de salto
(un scroll parallax vertical con jump\_bg\_2\_bamboo\_forest\_1.png,
jump\_bg\_3\_bamboo\_forest\_2.png. jump\_bg\_4\_mount.png. jump\_bg\_5\_pagoda.png.
jump\_bg\_6\_clouds.png. jump\_bg\_7\_lake.png. jump\_bg\_8\_mount\_fuji.png.
jump\_bg\_9\_sky.png. jump\_bg\_1\_stone\_statues.png), con una caña
de bambú a la izquierda y otra a la derecha (jump\_bg\_1\_bamboo.png),
y al personaje (jump\_ch\_ryoko.png, jump\_ch\_sho.png) entre ellas,
preparándose para empezar a saltar. Mientras se prepara, los pies
del personaje desprenden un aura (jump\_effect\_preparation.png). 

Se muestra el HUD muestra en la esquina inferior izquierda la cara
del personaje, y el la superior derecha el cronómetro, todos ellos
aún inactivos.

Se muestra en el centro de la pantalla un aviso de preparación/cuenta
atrás (Ready? Jump!). Juego Tras el inicio del juego se ponen en marcha
el cronómetro y el jugador puede iniciar los saltos pulsando el botón
de acción (el primer salto siempre es perfecto). 

Cada vez que el personaje alcanza el bambú hacia el que saltaba, se
muestra el marcador de salto, que muestra el ángulo/potencia de salto
(la animación del sprite del personaje y la cara del personaje en
el HUD cambian según el valor del ángulo de salto en ese momento). 

Si se pulsa el botón de acción el personaje rebota. Dependiendo del
ángulo, el salto será mejor o peor. Cuanto más arriba el salto será
mejor, pero si se pasa, el personaje se caerá.

Si el rebote es perfecto, se mostrará un efecto (jump\_effect\_wall\_kick.png)
en los pies del ninja.

Si el personaje cae desde menos de cierta altura (por ejemplo, dos
saltos), la caída no hace perder la prueba, sino que se puede continuar
saltando desde abajo del todo (pero el cronómetro no se resetea).
Si el personaje cae desde más arriba de esa altura, se pierde la prueba.
Resultado Si se llega arriba antes de que acabe el tiempo se supera
la prueba (el personaje salta y cae en equilibrio sobre el bambú).

Si se falla una pulsación y se cae desde muy alto o se pasa el tiempo
límite se falla la prueba (al pasar el tiempo el ninja también se
caería). 

Se mostrará una animación u otra en consecuencia del resultado de
la prueba.

Recursos (assets) Preparation: El personaje se prepara para hace el
primer salto desde el suelo. También sirve para repetir el salto desde
el suelo si el personaje se cae desde poca altura (el primer o segundo
rebotes como mucho).

Jump normal: Se usa cuando se consigue rebotar pero se le ha dado
muy pronto al botón.

Jump good: Se usa cuando se consigue rebotar y se espera suficiente
para que sea un buen salto.

Jump excellent: Se usa cuando se consigue rebotar y se hace un salto
perfecto (el primer salto desde el suelo usa también esta animación)

Reaching wall: Animación que se usa cuando el personaje está llegando
al bambú y se prepara para rebotar.

Charging jump: Se usa cuando se acaba de tocar la pared y se está
preparando el siguiente salto (en ese momento también se muestra el
marcador de salto)

Reaching top \& Win pose: Se usa cuando el personaje llega arriba
del todo.

Fail \& fall: Se usa cuando no se le da al botón a tiempo y se cae
el personaje jump\_ch\_ryoko.png (jump\_ch\_sho.png)

Efecto de “aura” que se muestra en los pies mientras se está preparando
para hacer el primer salto jump\_effect\_preparation.png

Efecto de “aura” que se muestra en los pies cuando se rebota contra
un bambú jump\_effect\_wall\_kick.png

Factor de velocidad aproximada de la capa parallax = -10.0f

Imagen que se usará en ambos bambú en los que el personaje rebota.

Se deberá trocear en tres partes: -El trozo inferior aparecerá abajo
del scroll. -El trozo medio se repetirá verticalmente a lo largo del
scroll parallax hasta completar la altura indicada. -El trozo superior
aparecerá arriba del todo del scroll.

Ver ejemplo funcionando con todas las capas en : https://github.com/jjhaggar/EjemploParallax
jump\_bg\_1\_bamboo.png

Factor de velocidad aproximada de la capa parallax = -10.0f

Estas estatuas de monjes aparecerán junto a uno de los bambús. Su
utilidad en el juego es meramente decorativa. jump\_bg\_1\_stone\_statues.png

Factor de velocidad aproximada de la capa parallax = -9.0f

Fondo más cercano al personaje. 

Al igual que sucede con el bambú, la imagen se dividirá en tres partes,
inferior, media y superior. La zona media se repite unas cuantas veces
(muchas menos que la del bambú sólo). jump\_bg\_2\_bamboo\_forest\_1.png

Factor de velocidad aproximada de la capa parallax = -5.5f

Segundo fondo más cercano al personaje. 

Al igual que sucede con el bambú, la imagen se dividirá en tres partes,
inferior, media y superior. La zona media se repite unas cuantas veces
(muchas menos que la del fondo más cercano al personaje). jump\_bg\_3\_bamboo\_forest\_2.png

Factor de velocidad aproximada de la capa parallax = -4.5f

Tercer fondo más cercano.

Montaña cercana, llena de árboles situada a la derecha de la pantalla
mientras se asciende. jump\_bg\_4\_mount.png

Factor de velocidad aproximada de la capa parallax = -4.0f

Cuarto fondo más cercano.

Pagoda situada tras la montaña cercana, también a la derecha de la
pantalla mientras se asciende. jump\_bg\_5\_pagoda.png

Factor de velocidad aproximada de la capa parallax = -2.0f

Nubes que en principio serán el quinto fondo más cercano. jump\_bg\_6\_clouds.png

Factor de velocidad aproximada de la capa parallax = -2.0f

El lago empataría con las nubes en lejanía. jump\_bg\_7\_lake.png

Factor de velocidad aproximada de la capa parallax = -1.8f

El monte Fuji sería (detrás del cielo) el segundo fondo más alejado.
jump\_bg\_8\_mount\_fuji.png

Factor de velocidad aproximada de la capa parallax = -0.5f

El cielo sería el fondo más alejado. Con un factor parallax de -0.5f,
casi no se movería. jump\_bg\_9\_sky.png

Este elemento del HUD no se usa en esta iteración. hud\_head\_jump.png

Este elemento del HUD servirá para indicar la potencia del siguiente
salto del personaje. Aparecerá brevemente cuando el personaje toque
el bambú para rebotar en él. hud\_angle\_indicator.png

Este elemento del HUD servirá para indicar la el ángulo actual del
salto. hud\_precision\_cursor.png

Música:

Nombre de archivo: trial\_jump.ogg

Loop point: - Sonidos:

Especificados en el epígrafe 4.2) Efectos de sonido


\subsubsection{Shuriken Throwing Trial (prueba de lanzamiento de shurikens)}

Esta prueba consiste en alcanzar con shurikens una serie de enemigos
(en principio hombres de paja). Para ello se puede apuntar moviendo
las manos a dereca o izquierda de la pantalla pulsando derecha e izquierda
en el D-pad y lanzar shurikens pulsando un botón. Hay que acabar con
todas las oleadas para superar la prueba.

Algo similar a la fase de bonus de Shinobi (© SEGA): http://youtu.be/Soyt3nTDfQ4?t=2m6s

Marcadores Descripción Cuenta atrás Al inicio de la fase se mostrará
una cuenta atrás (Ready? Go!). El jugador puede lanzar shurikens durante
la misma. Número de enemigos En la parte superior derecha se mostrará
un muñeco pequeño (hud\_head\_shuriken.png), a su derecha un signo
de multiplicar “x” y a la derecha de este el número de enemigos que
quedan. Marcador de shurikens Se muestra la cabeza del personaje y
a su lado el tantas imágenes de shuriken como shurikens puede lanzar
en ese momento (hud\_head\_shuriken.png). Cuando se lanza un shuriken
desaparece una de las imágenes de shuriken. A medida que pasa el tiempo
se van reponiendo (aproximadamente 1 por segundo).

¡Aviso! En la 1ª iteración este marcador no se utilizará. Manos personaje
La vertical sobre las manos indica a dónde estamos apuntando en ese
momento.

Controles: Hay que mover las manos a derecha o izquierda para apuntar
y pulsar un botón para disparar. - Tras iniciar el juego salen enemigos
al fondo y van recorriendo horizontalmente la pantalla y acercándose
de forma alternada. - El jugador puede desplazar las manos del personaje
a derecha e izquierda usando el D-Pad. - El jugador puede disparar
a los enemigos pulsando el botón de acción. Si un número de shurikens
(en principio uno) impacta en un enemigo, este es destruido. - Si
un número de enemigos (en principio uno) llegan a tocar al personaje
se pierde la partida.

La puntuación en la prueba la determinarán las siguientes mediciones:
- Precisión: Nº enemigos abatidos / Nº Shurikens lanzados (máximo
= 1). - Rapidez al eliminar los enemigos (una forma de medir esto
sería sumar el tiempo que se ha tardado en eliminar a todos los enemigos,
cuanto menor sea este tiempo, mejor puntuación).

Los niveles de dificultad implicarán diferencias en: - Número de enemigos.
- Velocidad de los enemigos. - Recorrido horizontal medio. - Número
de impactos que soportan los enemigos. - Número de “Vidas” (ataques
que puede encajar el personaje).

Victoria: Eliminar a todos los enemigos. Derrota: Consiguen alcanzar
al personaje cierto número de enemigos (en principio 1).

Descripción detallada Prepa-ración Se muestra el escenario de lanzamiento
de shurikens, y las manos del personaje. El HUD muestra en la esquina
superior derecha el número de enemigos que hay que abatir. Se muestra
en el centro de la pantalla un aviso de preparación/cuenta atrás (Ready?
Go!). Durante dicho aviso se puede mover las manos con el D-Pad y
lanzar shurikens con el botón de acción (pero al no haber empezado
la prueba, el lanzamiento de los mismos no contabilizará y no afectará
a la precisión). Juego Hay dos “carriles” horizontales, uno lejos
y otro a media distancia. Si no son abatidos por los shurikens los
enemigos bajarán primero por el que está lejos y recorrerán una parte
del “carril”, tras eso subirán de nuevo a los árboles y bajarán en
el “carril” que está a media distancia, recorrerán una parte de ese
carril y volverán a ascender a los árboles, tras eso caerá junto al
personaje y le quitará una “vida”. 

El jugador debe mover las manos del personaje a derecha o izquierda
para apuntar a los enemigos y pulsar el botón de acción para lanzar
shurikens. Los shurikens tienen una velocidad constante, con lo que
el jugador deberá calcular mentalmente en cada momento la trayectoria
de los enemigos para poder alcanzarlos al disparar. Resultado Si se
acaban las “vidas” del personaje (en principio una) o se elimina a
todos los enemigos se acaba la prueba.

Las manos desaparecerían bajando por la pantalla y según el resultado
se mostraría la animación de victoria o la de derrota.

Recursos (assets) Estos spritesheet sirven para:

El movimiento de las manos, tanto cuando se está quieto en un mismo
sitio o cuando se está desplazando a derecha e izquierda (por ahora
todo es el mismo frame, el 3, pero más adelante se pueden hacer animaciones
simples para los tres casos).

El lanzamiento de un shuriken (usando los tres frames) shuriken\_sho\_hands.png
(shuriken\_ryoko\_hands.png)

Este spritesheet sirve para animar el movimiento del shuriken lanzado
al alejarse.

Se hace así en lugar de usar una sóla imagen y rotarla mediante programación
para que la animación encaje con el espíritu “oldschool” del resto
del juego.

shuriken\_shuriken.png

Animación de victoria. shuriken\_sho\_win.png (shuriken\_ryoko\_win.png)

Animación de derrota. shuriken\_sho\_lose.png (shuriken\_sho\_lose.png)

Enemigos a lo lejos.

Los dos fotogramas de la derecha son para cuando aparece el muñeco
desde arriba. También sirve para cuando desaparece nuevamente hacia
arriba.

El primer fotograma es para cuando se desplaza lateralmente (rotar
para el desplazamiento al otro lado)

shuriken\_strawman\_1.png

Enemigos a media distancia.

Al igual que con el spritesheet anterior, los dos fotogramas de la
derecha son para cuando aparece el muñeco desde arriba y para cuando
desaparece nuevamente hacia arriba. Y del mismo modo el primer fotograma
es para cuando se desplazan lateralmente (rotar para el desplazamiento
al otro lado)

shuriken\_strawman\_2.png

Enemigo cerca del personaje.

Si puedes leer el letrero con el kanji “muere” que hay en su pecho
es que estás demasiado cerca.

Aparece cuando algún enemigo ha conseguido recorrer los dos caminos
(primero el segmento a lo lejos y luego el segmento a media distancia)
sin ser abatido. shuriken\_strawman\_3.png

Fondo de la fase. Ahora mismo es fijo y en una sola capa. Más adelante
probablemente se divida en dos capas como mínimo.

shuriken\_background.png

Esta imagen contiene varios elementos del HUD.

Se usará el pequeño muñeco para crear un contador para los enemigos
que quedan por destruir. La imagen irá junto a un número que muestre
ese número.

En esta iteración no se usará las imágenes de las cabezas ni las de
los shurikens.

hud\_head\_shuriken.png

Música:

Nombre de archivo: trial\_cut.ogg

Loop point: - Sonidos:

Especificados en el epígrafe 4.2) Efectos de sonido


\subsection{Pause Screen (pantalla de pausa)}

Llegamos a la pantalla de pausa desde cualquier prueba/trial pulsando
el \textquotedbl{}botón Ouya\textquotedbl{}

Se detienen la música y las animaciones del juego, y se oscurece la
pantalla bajo el “cartel” de pausa (menu\_pause\_bamboo\_frame.png),
mostrándose las opciones \textquotedbl{}Continuar\textquotedbl{} (marcada
por defecto), “Opciones” y \textquotedbl{}Salir\textquotedbl{}.

Si se pulsan el botón “HOME\textquotedbl{} o el botón “A” en cualquier
sitio, o si se pulsan los botones “O”, “U” o “Y” teniendo seleccionada
la opción \textquotedbl{}Continuar\textquotedbl{} se reanuda el juego.
Si se pulsan los botones “O”, “U” o “Y” teniendo seleccionada la opción
\textquotedbl{}Opciones\textquotedbl{} se pasa al \textquotedbl{}menú
opciones\textquotedbl{}. Si se pulsan los botones “O”, “U” o “Y” teniendo
seleccionada la opción \textquotedbl{}Salir\textquotedbl{} se presenta
un diálogo de confirmación “Sí”/”No” - Si se elige “No”, se vuelve
a la pantalla de pause original con las 3 opciones. - Si se elige
“Sí”, se pasa al \textquotedbl{}menú principal\textquotedbl{}.


\subsection{Lose / Continue Screen (pantalla de derrota / continuación)}

Se muestra un fondo (results\_lose\_background.png) y una imagen con
el ninja derrotado (results\_lose\_ch\_ryoko.png, results\_lose\_ch\_sho.png)
al tiempo que se escucha una música de derrota (y/o la frase \textquotedbl{}You
Lose!/You Lost!\textquotedbl{}... ¿en japonés?). Acto seguido se muestra
el texto \textquotedbl{}Continúe?\textquotedbl{} y empieza una cuenta
atrás de 10 segundos (en el caso de estar jugando en el nivel de dificultad
\textquotedbl{}difícil\textquotedbl{}, se muestran los intentos/vidas/continuaciones
que le quedan al jugador) hay dos opciones: \textquotedbl{}Yes\textquotedbl{}
(seleccionada por defecto) y \textquotedbl{}No\textquotedbl{}.

Si el jugador pulsa sobre \textquotedbl{}Yes\textquotedbl{} (y le
quedan continuaciones) pasa a la pantalla de mapa y continúa por la
misma prueba en la que falló (en este caso no saldrá el \textquotedbl{}how
to play\textquotedbl{} al pulsar un botón, sino directamente la prueba).
Si el jugador pulsa sobre \textquotedbl{}No\textquotedbl{}, o se acaba
el tiempo de la cuenta atrás, o no le quedan \textquotedbl{}continues\textquotedbl{},
se pone la pantalla en gris, se muestra la frase \textquotedbl{}Game
over\textquotedbl{} y tras un fundido a negro se vuelve a la pantalla
de inicio.


\subsection{Win / Results Screen (pantalla de victoria / resultados)}

Se muestra un pergamino en el centro (results\_win\_scroll.png) con
la imagen de la prueba (results\_win\_drawings.png) y a la izquierda
(o a cada lado si están jugando ambos jugadores) el personaje en pose
de victoria (results\_win\_ch\_ryoko.png, results\_win\_ch\_sho.png)
y en su mismo lado se muestran sus puntos en la prueba y una imagen
de un sello con su calificación (results\_win\_stamp\_ranking.png).
Las calificaciones podrían ser, por ejemplo: Novice, Ninja, Ninja
Master, Grand Master. 

Los puntos se van sumando poco a poco a la puntuación global, que
se mostrará abajo, emitiendo al mismo tiempo un sonido característico
de \textquotedbl{}incremento de puntos\textquotedbl{} (menu\_points\_sum.ogg). 

Si se pulsa un botón se interrumpe la animación de los números que
estaban \textquotedbl{}sumando poco a poco\textquotedbl{} y se suman
del todo (por ejemplo, so se estaban sumando los puntos del “tiempo”,
se terminan de sumar todos esos puntos y se pasa a sumar los puntos
de “combo”). Si se pulsa start se omiten todas las animaciones y se
muestra el resultado de todas las sumas en la puntuación global, al
mostrarse este resultado sse escucha el sonido de “fin de suma de
puntos” (menu\_points\_total.ogg).

Esta animación y sonidos no es necesario añadirlos en esta iteración,
pero se muestra un ejemplo similar a lo que se pretende en el siguiente
enlace: http://youtu.be/5NLy\_ZZB\_vk?t=1m50s (se suma el tiempo y
luego el perfect).

Si se pulsa cualquier botón mientras se está mostrando el resultado
final y quedan pruebas por superar se pasa a la pantalla del mapa.
Si se pulsa cualquier botón mientras y se han superado todas la pruebas
se pasa a la pantalla de \textquotedbl{}ending\textquotedbl{}.


\subsection{Ending Screen (pantalla de final de juego)}

El personaje le muestra al maestro los pergaminos de todas las pruebas
que ha superado y este muestra su aprobación.

Tras eso se muestra una escena (probablemente cómica) del personaje
tras ser nombrado ninja en su clan. Las escenas constarán de texto
y de animaciones simples, hechas en su mayoría mediante desplazamiento
de capas superpuestas.

Habrá una escena de ending diferente para cada personaje. En posteriores
iteraciones aparte de los dos anteriores también habrá uno para ambos,
y los tres variarán en función de la dificultad.

Las historias y storyboards están aún por determinar.

Si se pulsa un botón cualquiera no pasa nada. Si se pulsa el botón
start aparece durante tres segundos un texto pequeño con el mensaje
\textquotedbl{}pulsa start de nuevo para omitir\textquotedbl{}, si
se pulsa start de nuevo antes de que desaparezca el mensaje se interrumpe
el \textquotedbl{}ending\textquotedbl{} y se pasa a la \textquotedbl{}pantalla
de créditos\textquotedbl{}.


\subsection{Credits Screen (pantalla de créditos)}

Se muestra sobre un fondo (ending\_credits\_background.png) un listado
con toda la gente que ha hecho posible el proyecto junto con imágenes
cómicas de ninjas relacionadas con ellos o su trabajo (ending\_credits\_categories.png),
incluyendo colectivos como la comunidad de AndEngine, etc. Se mostrarán
también sus logotipos (ending\_credits\_logo\_estudioevergreen.png,
ending\_credits\_logo\_andengine.png).

Está por determinar tanto el listado como el modo de irlo mostrando.
Dos posibles opciones:

1) Mostrarlo como suele hacerse en los “créditos de las películas”.
Un listado de texto y logotipos que van subiendo por la pantalla de
forma continua.

2) Mostrarlo \textquotedbl{}a pantallazos\textquotedbl{}, haciendo
aparecer y desvanecer las letras cada cierto tiempo para mostrar el
texto y los logotipos.

Si se pulsa un botón cualquiera no pasa nada. Si se pulsa el botón
start aparece durante tres segundos un texto pequeño con el mensaje
\textquotedbl{}pulsa start de nuevo para omitir\textquotedbl{}, si
se pulsa start de nuevo antes de que desaparezca el mensaje se interrumpe
la \textquotedbl{}pantalla de créditos\textquotedbl{} y se pasa a
la \textquotedbl{}pantalla de récords\textquotedbl{}, destacando el
resultado actual (incluso aunque no se esté entre los primeros de
la lista).


\subsection{Loading Screen (pantalla de carga) {*}}

Esta pantalla sólo se realizará si es estrictamente necesaria. Si
por el contrario es posible cargar todos los assets en la \textquotedbl{}Splash
screen\textquotedbl{} sin perjuicio de la experiencia de juego, la
pantalla de carga no se creará.

La pantalla de carga mostraría el texto \textquotedbl{}Now loading\textquotedbl{},
e incluiría un minijuego del tipo \textquotedbl{}Mover un gato ninja
a derecha a izquierda por la pantalla\textquotedbl{} o \textquotedbl{}rotar
un shuriken\textquotedbl{} pulsando derecha o izquierda. Algo que
amenice la carga pero cuyo sprite sea pequeño y tenga dos fotogramas
como máximo.


\section{Audio}

Los efectos de sonido y las músicas estarán comprimidos en formato
Ogg Vorbis, a una frecuencia de muestreo de 44.1 kHz (o quizá de 48
kHz) y con un bitrate de 192 kbits/s (calidad -q6), o quizá de 320
kbits/s (calidad -q9). 

Sería deseable utilizar formato de música tracker (\textquotedbl{}.it\textquotedbl{}
o \textquotedbl{}.mod\textquotedbl{}) en lugar de \textquotedbl{}.ogg\textquotedbl{},
por su tamaño excepcionalmente reducido y su diseño específico para
crear diversos bucles en las músicas. Sin embargo esto entraña tres
problemas: 1. No sabemos si con ello ahorraríamos recursos o todo
lo contrario (necesitaríamos hacer antes pruebas de rendimiento con
músicas \textquotedbl{}.ogg\textquotedbl{} vs músicas \textquotedbl{}.it\textquotedbl{}
para comprobarlo). 2. Habría que rehacer las músicas desde cero (por
ejemplo con OpenMPT). No hay métodos para lograr una conversión de
música en formato de audio estándar a música en formato tracker de
forma automatizada. 3. Ciertos efectos en la música se perderían,
o serían bastante complicados de hacer. Pese a lo anterior, no se
descarta que en próximas iteraciones se hagan pruebas para que las
músicas estén en este formato.

Documentación relacionada: Frecuencia de muestreo: http://en.wikipedia.org/wiki/Sampling\_rate\#Audio
Bitrate: http://en.wikipedia.org/wiki/Bit\_rate\#Audio Música tracker:
http://en.wikipedia.org/wiki/Tracker\_(music\_software)

A continuación (epígrafes 4.1 y 4.2) se listan las músicas y los efectos
de sonido, indicando su uso.


\subsection{Músicas}

Estarán en el subdirectorio \textquotedbl{}assets/music\textquotedbl{},
y salvo indicar lo contrario tendrán extensión \textquotedbl{}ogg\textquotedbl{}.

Nombre Descripción / Pantallas en las que saldrá ¿Hecha? Duración
Loop intro1 Saldrá en la primera presentación Sí

menu Pantalla de menú principal, opciones, logros, info. pers. y selec.
pers. No

records Pantalla de récords. No

intro2 Segunda presentación (tras elegir personaje) No

map Pantalla de mapa No

trial\_run Prueba de carrera En proceso

trial\_cut Prueba de corte En proceso

trial\_jump Prueba de salto de pared en pared Sí

trial\_throw Prueba de lanzamiento de shuriken Sí

result\_win Pantalla de victoria / resultado de la prueba

result\_lose Pantalla de derrota / continuación En proceso

game\_over Pantalla de “Juego Terminado” (cuando se pierde y no se
continúa) En proceso

ending Pantalla de final de juego (quizás haga falta más de una música)
En proceso

credits Pantalla con los créditos del juego. En proceso


\subsection{Efectos de sonido}

Estarán en el subdirectorio \textquotedbl{}assets/sounds\textquotedbl{},
y salvo indicar lo contrario tendrán extensión \textquotedbl{}ogg\textquotedbl{}.

Nombre Descripción / Pantallas en que saldrá menu\_logo\_madgear Sonido
de engranajes mientras sale nuestro logo menu\_intro1 Saldrá al final
de la primera presentación, probablemente un sonido metálico, de un
shuriken o similar menu\_focus Pantallas de menú principal, opciones
y logros. Se escuchará al cambiar el foco de una opción a otra menu\_activate
Pantallas de menú principal, opciones y logros. Se escuchará al elegir
una opción menu\_back Pantallas de menú principal, opciones y logros.
Se escuchará al volver atrás menu\_points\_sum Sonido de los puntos
al sumarse menu\_points\_total Sonido de los puntos al concluir la
suma menu\_achievement Sonido de cuando se consigue un logro menu\_rank
Sonido de cuando te dan la calificación obtenida en la prueba judge\_9
Voces de la cuenta atrás judge\_8 Voces de la cuenta atrás judge\_7
Voces de la cuenta atrás judge\_6 Voces de la cuenta atrás judge\_5
Voces de la cuenta atrás judge\_4 Voces de la cuenta atrás judge\_3
Voces de la cuenta atrás judge\_2 Voces de la cuenta atrás judge\_1
Voces de la cuenta atrás judge\_ready Voces de la cuenta atrás judge\_go
Voces de la cuenta atrás judge\_you\_lose Voz de \textquotedbl{}You
lose/failed\textquotedbl{} judge\_you\_win Voz de \textquotedbl{}You
win\textquotedbl{} judge\_game\_over Voz de \textquotedbl{}Game Over\textquotedbl{}
judge\_good Voz que comunica el resultado de la prueba judge\_great
Voz que comunica el resultado de la prueba judge\_excellent Voz que
comunica el resultado de la prueba trial\_run\_tap1 Sonido de pasos
al correr (lento) trial\_run\_tap2 Sonido de pasos al correr (normal)
trial\_run\_tap3 Sonido de pasos al correr (rápido) trial\_run\_wind\_1\_start
Sonido al arrancar a velocidad máxima trial\_run\_wind\_2\_running
Sonido al cortar el viento mientras se corre a velocidad máxima trial\_run\_wind\_3\_end
Sonido al llegar a velocidad máxima trial\_cut\_whoosh1 Sonidos de
la katana al cortar el viento trial\_cut\_whoosh2 Sonidos de la katana
al cortar el viento trial\_cut\_whoosh3 Sonidos de la katana al cortar
el viento trial\_cut\_katana\_cut1 Sonidos de corte trial\_cut\_katana\_cut2
Sonidos de corte trial\_cut\_katana\_cut3 Sonidos de corte trial\_cut\_candle\_blow\_out
Sonidos de vela al apagarse trial\_cut\_candle\_showing\_cut Sonidos
de vela al mostrar el corte trial\_cut\_candle\_wobble Sonidos de
vela al tambalearse a punto de caerse trial\_cut\_candle\_wobble\_thud
Sonidos de vela al tambalearse y caerse trial\_cut\_candle\_thud Sonidos
de vela al caer y tocar el suelo trial\_cut\_eyes\_zoom Sonido de
tensión cuando ha realizado el corte (al abrir los ojos) trial\_jump\_tap1
Sonido al rebotar en un bambú trial\_jump\_tap2 Sonido al rebotar
en un bambú trial\_jump\_whoosh1 Sonido al saltar (salto normal) trial\_jump\_whoosh2
Sonido al saltar (salto bueno) trial\_jump\_whoosh3 Sonido al saltar
(salto excelente) trial\_jump\_reach Sonido al saltar cuando se llega
arriba del todo trial\_jump\_wobble Sonido de cuando se falla el salto
y el personaje intenta no caerse trial\_jump\_slip Sonido de cuando
se falla el salto y el personaje resbala del bambú trial\_jump\_fall
Sonido de mientras el personaje cae trial\_jump\_thud Sonido de cuando
el personaje se golpea contra el suelo tras la caída trial\_shuriken\_strawman\_descend
Sonido de los muñecos de paja al caer de entre las ramas trial\_shuriken\_strawman\_move
Sonido de los muñecos de paja al moverse a izquierda o derecha trial\_shuriken\_strawman\_ascend
Sonido de los muñecos de paja al subir de nuevo a las ramas trial\_shuriken\_strawman\_hit
Sonido de los muñecos de paja al recibir un golpe trial\_shuriken\_strawman\_destroyed
Sonido de los muñecos de paja al ser destruídos trial\_shuriken\_throwing
Sonido del shuriken al surcar el aire effect\_eye\_gleam Sonido de
brillo en un ojo del personaje effect\_sweat\_drop Sonido de gota
de sudor resbalando por la frente effect\_master\_hit Sonido del maestro
al golpear a los protagonistas sho\_run\_charge Voz de Sho en el Trial
Run. Preparádose para salir corriendo (¡Hmmm!) sho\_run\_start Trial
Run. Voz al empezar a correr a toda velocidad. sho\_run\_win Trial
Run. Voz de victoria (¡Hum!) sho\_run\_lose Trial Run. Voz de derrota
(¡Ie!) sho\_cut\_cut Trial Cut. Voz de cuando hace un corte (¡Ha!)
sho\_cut\_win Trial Cut. Voz de victoria (¡Hum!) sho\_cut\_lose Trial
Cut. Voz de derrota (¡Ie!) sho\_jump\_charge Trial Jump. Voz de Sho
preparándose para el primer salto sho\_jump\_hop Trial Jump. Voz de
salto potente. sho\_jump\_fall Trial Jump. Grito cuando se cae desde
bastante alto sho\_jump\_win Trial Jump. Frase de victoria (¡Oh!)
sho\_jump\_lose Trial Jump. Frase de derrota (¡Itai!) sho\_shuriken\_throw
Trial Shuriken. Voz de cuando lanza un shuriken (¡Ha!). Se escucha
de forma aleatoria, no siempre que se lance un shuriken sho\_shuriken\_win
Trial Shuriken. Frase de victoria (¡Yosh!) sho\_shuriken\_lose Trial
Shuriken. Frase de derrota (¡Argh!) sho\_menu\_game\_over Voz cuando
se pasa de la pantalla de “continue” a la de “game over” (Argh) sho\_menu\_continue
Voz cuando se continua (¡Ha!) ryoko\_run\_charge Voz de Ryoko en el
Trial Run. Preparádose para salir corriendo (¡Hmmm!) ryoko\_run\_start
Trial Run. Voz al empezar a correr a toda velocidad. ryoko\_run\_win
Trial Run. Voz de victoria (¡Hum!) ryoko\_run\_lose Trial Run. Voz
de derrota (¡Ie!) ryoko\_cut\_cut Trial Cut. Voz de cuando hace un
corte (¡Ha!) ryoko\_cut\_win Trial Cut. Voz de victoria (¡Hum!) ryoko\_cut\_lose
Trial Cut. Voz de derrota (¡Ie!) ryoko\_jump\_charge Trial Jump. Voz
de ryoko preparándose para el primer salto ryoko\_jump\_hop Trial
Jump. Voz de salto potente. ryoko\_jump\_fall Trial Jump. Grito cuando
se cae desde bastante alto ryoko\_jump\_win Trial Jump. Frase de victoria
(¡Hah!) ryoko\_jump\_lose Trial Jump. Frase de derrota (¡Itai!) ryoko\_shuriken\_throw
Trial Shuriken. Voz de cuando lanza un shuriken (¡Ha!). Se escucha
de forma aleatoria, no siempre que se lance un shuriken ryoko\_shuriken\_win
Trial Shuriken. Frase de victoria (¡Yatta!) ryoko\_shuriken\_lose
Trial Shuriken. Frase de derrota (¡Argh!) ryoko\_menu\_game\_over
Voz cuando se pasa de la pantalla de “continue” a la de “game over”
(Argh) ryoko\_menu\_continue Voz cuando se continua (¡Ha!)

¡Aviso! Las frases y voces de Ryoko y Sho están aún por definir. Se
han puesto ejemplos como placeholders, pero las definitivas se decidirán
una vez que se tenga elaborada la historia y se consulte con un traductor
de japonés.


\section{Gráficos}

Estarán en el subdirectorio \textquotedbl{}assets/gfx\textquotedbl{},
y las extensiones serán \textquotedbl{}png\textquotedbl{}, \textquotedbl{}jpg\textquotedbl{}
y \textquotedbl{}svg\textquotedbl{}. En su interior a su vez habrá
varios subdirectorios específicos.

Subdirectorio Descripción

En raiz ( assets/gfx/ ) se incluiran las imágenes no recogidas en
los otros subdirectorios, como las de los “character profile” y del
“how to play”. splash Contendrá los logos que aparecerán al arrancar
el juego. intro\_1 Contendrá los gráficos de la pantalla de presentación
que se muestra tras el “splash” (salvo los relativos a los trials).
intro\_2 Gráficos de la intro que se muestra tras elegir personaje
y dificultad. menus Gráficos de los menús: “Principal”, “logros”,
“mapa”, “opciones”, “pausa” y “selección de personaje y dificultad”.
hud Gráficos para los HUDs de los trials. trial\_cut Gráficos para
la prueba de corte. trial\_jump Gráficos para la prueba de salto.
trial\_run Gráficos para la prueba de correr. trial\_shuriken Gráficos
para la prueba de lanzar shurikens. results Gráficos de las pantallas
de “victoria”, “derrota (continue)”, “records” y “game over”. endings
Gráficos de los diferentes endings y de la pantalla de créditos.

Independientemente de que estén distribuidos en subdirectorios, cada
archivo tendrá un nombre que identifique su lugar de procedencia,
por si en un futuro se deciden reubicar al directorio raiz. Ejemplos:
menu\_achievement\_success\_stamp.png menu\_achievement\_icons\_big.png
menu\_achievement\_icons\_small.png


\section{Texto}

Todo el texto que se muestre en el juego se encontrará en los archivos
string.xml designados para ello tal y como se indica en página de
Android Developer. http://developer.android.com/guide/topics/resources/string-resource.html

El texto deberá estar disponible como mínimo en dos idiomas, inglés
y español, aunque se intentará que esté disponible en cuantos más
idiomas mejor. La disposición de los archivos string.xml será la siguiente:

Proyecto/ res/ values/ strings.xml values-es/ strings.xml

Se añadirá una carpeta y archivo string.xml por cada idioma extra.
Por ejemplo si quisiéramos añadir el francés, crearíamos en el proyecto
la carpeta res/values-fr, y dentro de ella el archivo strings.xml,
tal y como se indica a continuación:

values-fr/ strings.xml Justo como se describe en: http://developer.android.com/training/basics/supporting-devices/languages.html

Para facilitar las labor de los traductores, las cadenas que lo requieran
deben proveer contexto suficiente mediante un comentario, que se situará
delante de la(s) cadena(s) en cuestión. Por ejemplo:

<!-- Main menu buttons --> <string name=\textquotedbl{}start\_button\textquotedbl{}>Start!</string>
<string name=\textquotedbl{}menu\_button\textquotedbl{}>Menu</string>
<string name=\textquotedbl{}options\_button\textquotedbl{}>Options</string>

Todo ello tal y como se describe en: http://developer.android.com/distribute/googleplay/publish/localizing.html\#strings

Se han probado diferentes aplicaciones para una mayor comodidad en
la traducción de los archivos strings.xml El programa que se propone
para efectuar las traducciones, por su funcionalidad y sencillez de
uso, es el TM-Database: http://yehongmei.narod.ru/

La versión 1.82 http://yehongmei.narod.ru/TM-182.htm funciona correctamente
bajo Windows 7 tras aplicar un parche que corrige un problema que
hace que el programa se cierre al poco tiempo de iniciar. A continuación
dos enlaces (autorizados por los creadores) con dicho parche: Mediafire
- Mega 


\section{Calendario}

A concretar.
\end{document}
